\chapter{Cryo-CARE}
%{{{

CARE (Content-Aware image REstoration) methods leverage available
knowledge about the data at hand ought to yield superior restoration
results \cite{weigert2018content}. Concretely, Cryo-CARE
\cite{buchholz2019cryo} is an implementation of Noise2Noise (N2N)
\cite{lehtinen2018noise2noise}.

N2N is a ``supervised'' learning method for denoising where an
autoencoder neural network with skip connections (a U-Net) is trained
on pairs of noisy images. However, unlike clasical supervised
denoising deep-learning based models, that usually implement
\cite{lehtinen2018noise2noise}
\begin{equation}
  \underset{\theta}{\operatorname{arg\,min}} \, \sum_j L \big(f_\theta(\hat{\mathbf X}_j^{(1)}), {\mathbf X}_j\big)
\end{equation}
\begin{equation}
  \underset{\theta}{\operatorname{arg\,min}} \, \sum_j L \big(f_\theta(\{\hat{\mathbf X}\}_j), \{{\mathbf X}\}_j\big)
\end{equation}

where $\{(\hat{\mathbf{X}}, \mathbf{X})\}_j\}$ is the training
dataset, and $L$ is a given lost function such as the MSE, N2N solve

where $\{(\hat{\mathbf X}_j^{(1)}, {\mathbf X}_j)\}_{j=1}^M$ is the training
dataset, and $L$ is a given lost function such as the MSE, N2N solves
\begin{equation}
  \underset{\theta}{\operatorname{arg\,min}} \, \sum_j L \big(f_\theta(\hat{\mathbf X}_j^{(1)}), {\mathbf X}_j^{(2)}\big).
\end{equation}
In other words, given two noisy versions
$\{\hat{\mathbf Y}^{(1)}, \hat{\mathbf Y}^{(2)}\}$ of the same (clean)
volume ${\mathbf Y}$, N2N learns to infeer a denoised volume
\begin{equation}
  \tilde{\mathbf Y}=\frac{1}{2}\big(f_\theta(\hat{\mathbf Y}^{(1)})+f_\theta(\hat{\mathbf Y}^{(2)})\big)\approx{\mathbf Y}.
\end{equation}
Obviously, better approximations to ${\mathbf Y}$ will be obtained
having more noisy instances, after averaing all the denoised volumes.

%}}}
