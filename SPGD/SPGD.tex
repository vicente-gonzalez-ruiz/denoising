\chapter{Structure-Preserving Gaussian Denoising (SPGD)}
%{{{

\begin{figure}
  \centering
  \includegraphics[width=\textwidth]{yes_OF_images}
  \caption{SPGD along the $\mathrm{Z}$-dimension. Five consecutive
    slices (or $\mathrm{XY}$-planes) are involved in the filtering of
    the slice $\hat{\mathbf{X}}_z$. Four displacements fields
    (represented by arrows) determined by the OF estimator are used
    for the alignment of the structures found in the slices
    $\hat{\mathbf{X}}_{z-2}, \hat{\mathbf{X}}_{z-1},
    \hat{\mathbf{X}}_{z+1},$ and $\hat{\mathbf{X}}_{z+2}$ with respect
    to those present in $\hat{\mathbf{X}}_{z-2}$, hence producing
    OF-compensated slices. The Gaussian filtering along the
    $\text{Z}$-dimension (dotted line) is then applied to the
    OF-compensated slices.\label{fig:SPGD}}
\end{figure}

In its 3D version, SPGD is based in 3D Gaussian filtering (see
Eq.~\ref{eq:3D_GF}), but the noised volume $\hat{\mathbf{X}}$ is
slice-wise 2D-warped (see Fig. \ref{fig:SPGD}) in the 3 space
dimensions (see Fig. \ref{fig:3D_GF}), resulting (compared to GF) in a
reduction of the bluring of the structures detected by a 2D OF
(Optical Flow) estimator \cite{gonzalez2023structure}. This idea can
be expressed as
\begin{equation}
  \tilde{\mathbf{X}}  = R_\text{X}\Big(R_\text{Y}\big(R_\text{Z}(\hat{\mathbf{X}})*^{(\text{Z})}{\mathbf h}\big)*^{(\text{Y})}{\mathbf h}\Big)*^{(\text{X})}{\mathbf h},
    \label{eq:SDPG}
\end{equation}
where
\begin{equation*}
  \begin{array}{rclll}
    R_\text{Z}(\mathbf{X}) & = & \big\{ \{ \overset{z'\rightarrow z}p(\mathbf{X}_{z',:,:})~:~\overset{z'\rightarrow z}p(\mathbf{X}_{z',:,:})\approx\mathbf{X}_{z,:,:} & \\ & & \text{for}~z'=z-\lceil\mathbf{h}.\mathsf{size}/2\rceil,\cdots,z+\lceil\mathbf{h}.\mathsf{size}/2\rceil\} & \\ & & \text{for}~z=0,1,\cdots,\mathbf{X}.\mathsf{shape}_0-1\big\}, \\
    R_\text{Y}(\mathbf{X}) & = & \big\{ \{ \overset{y'\rightarrow y}p(\mathbf{X}_{:,y',:})~:~\overset{y'\rightarrow y}p(\mathbf{X}_{:,y',:}\approx\mathbf{X}_{[:,y,:]} & \\ & & \text{for}~y'=y-\lceil\mathbf{h}.\mathsf{size}/2\rceil,\cdots,y+\lceil\mathbf{h}.\mathsf{size}/2\rceil\} & \\ & & \text{for}~y=0,1,\cdots,\mathbf{X}.\mathsf{shape}_1-1\big\},~\text{and} \\
    R_\text{X}(\mathbf{X}) & = & \big\{ \{ \overset{x'\rightarrow x}p(\mathbf{X}_{:,:,x'})~:~\overset{x'\rightarrow x}p(\mathbf{X}_{:,:,x'}\approx\mathbf{X}_{:,:,x} & \\ & & \text{for}~x'=x-\lceil\mathbf{h}.\mathsf{size}/2\rceil,\cdots,x+\lceil\mathbf{h}.\mathsf{size}/2\rceil\} & \\ & & \text{for}~x=0,1,\cdots,\mathbf{X}.\mathsf{shape}_2-1\big\}
    \end{array}
\end{equation*}
are the slice-wise warped volumes. For example,
$\overset{x'\rightarrow x}p(\mathbf{X}_{:,:,x'})$ represents the
projection of the slice at slice-index $x'$ fulfilling that
$\overset{x'\rightarrow
  x}p({\mathbf{X}})\approx{\mathbf{X}}_{:,:,x}$. Notice that, for
each possible offset in $\text{Z}$, $\text{Y}$, and $\text{X}$, a
different set of warped 2D slices must be computed.

\begin{comment}
%{{{

\begin{figure}
\noindent $\mathsf{3DSPGF}(\hat{\mathbf{X}}, \mathbf{h}, w)$: $\rightarrow\tilde{\mathbf{X}}$
\vspace{-1ex}
\begin{enumerate}
  \setlength{\itemsep}{0pt}
\item [1.] $\tilde{\mathbf{X}}\leftarrow\mathsf{zeros\_like}(\hat{\mathbf{X}})$
\item [2.] $\mathsf{for}~z~\mathsf{in}~\mathsf{range}(\hat{\textbf{X}}.\mathsf{shape}_0),~\mathsf{run}$:  \hfill $\mathtt{/*~Filtering~in~the~Z~direction~*/}$
  \begin{enumerate}
  \item [1.] $\mathsf{for}~h~\mathsf{in~range}(\mathbf{h}.\mathsf{size}),~\mathsf{run}$:
    \begin{enumerate}
    \item [1.] $R_\text{Z}\leftarrow\href{https://docs.opencv.org/3.4/da/d54/group__imgproc__transform.html#gab75ef31ce5cdfb5c44b6da5f3b908ea4}{\mathsf{warp}}(\hat{\mathbf{X}}_{z+h,:,:}, \href{https://docs.opencv.org/3.4/dc/d6b/group__video__track.html#ga5d10ebbd59fe09c5f650289ec0ece5af}{\mathsf{flow}}(\hat{\mathbf{X}}_{z+h,:,:}, \hat{\mathbf{X}}_{z,:,:}))$  
    \item [2.] $\tilde{\mathbf{X}}_{z,:,:}\leftarrow\tilde{\mathbf{X}}_{z,:,:}+R_\text{Z}\mathbf{h}_{h}$
    \end{enumerate}
  \end{enumerate}
\item [3.] $\mathsf{for}~y~\mathsf{in}~\mathsf{range}(\hat{\textbf{X}}.\mathsf{shape}_1),~\mathsf{run}$:  \hfill $\mathtt{/*~Filtering~in~the~Y~direction~*/}$
  \begin{enumerate}
  \item [1.] $\mathsf{for}~k~\mathsf{in~range}(\mathbf{h}.\mathsf{size}),~\mathsf{run}$:
    \begin{enumerate}
    \item [1.] $R_\text{Y}\leftarrow\href{https://docs.opencv.org/3.4/da/d54/group__imgproc__transform.html#gab75ef31ce5cdfb5c44b6da5f3b908ea4}{\mathsf{warp}}(\hat{\mathbf{X}}_{:,y+h,:}, \href{https://docs.opencv.org/3.4/dc/d6b/group__video__track.html#ga5d10ebbd59fe09c5f650289ec0ece5af}{\mathsf{flow}}(\hat{\mathbf{X}}_{:,y+h,:}, \hat{\mathbf{X}}_{:,y,:}))$
    \item [2.] $\tilde{\mathbf{X}}_{:y,:}\leftarrow\tilde{\mathbf{X}}_{:,y,:}+R_\text{Y}\mathbf{h}_{h}$
    \end{enumerate}
  \end{enumerate}
\item [4.] $\mathsf{for}~x~\mathsf{in}~\mathsf{range}(\hat{\textbf{X}}.\mathsf{shape}_2),~\mathsf{run}$:  \hfill $\mathtt{/*~Filtering~in~the~X~direction~*/}$
  \begin{enumerate}
  \item [1.] $\mathsf{for}~k~\mathsf{in~range}(\mathbf{h}.\mathsf{size}),~\mathsf{run}$:
    \begin{enumerate}
    \item [1.] $R_\text{X}\leftarrow\href{https://docs.opencv.org/3.4/da/d54/group__imgproc__transform.html#gab75ef31ce5cdfb5c44b6da5f3b908ea4}{\mathsf{warp}}(\hat{\mathbf{X}}_{:,:,x+h}, \href{https://docs.opencv.org/3.4/dc/d6b/group__video__track.html#ga5d10ebbd59fe09c5f650289ec0ece5af}{\mathsf{flow}}(\hat{\mathbf{X}}_{:,:,x+h}, \hat{\mathbf{X}}_{:,:,x}))$
    \item [2.] $\tilde{\mathbf{X}}_{:,:,x}\leftarrow\tilde{\mathbf{X}}_{:,:,x}+R_\text{X}\mathbf{h}_{h}$
    \end{enumerate}
  \end{enumerate}
\end{enumerate}
%\end{myquote}
\caption{Python pseudo-code of 3DSPGF using periodic signal extension.}
\label{fig:3DSPGF_imple}
\end{figure}

The 2D case of SPGD operates only over the Y and X axis, and works by
aligning 1D lines instead of 2D frames before using GF. In the current
implementation we are using the same 2D OF estimator than for the 3D
case, splitting the image into overlaping 2D slices of several
consecutive rows of columns (depending on if are filtering in the Y or
the X direction) and computing the OF between these slices. Then, for
each slice, we use only the line of the center to generate the
OF-warped line.

%}}}
\end{comment}

Compared with GF, the OF estimator requires some new parameters
\cite{farneback2003two} that can affect to the denoised volumes:
\begin{enumerate}
\item The \textbf{Averaging Window Size (AWS)} that controls the
  blurring of the generated motion field, and therefore, the blurring
  of the denoised volume. The larger the size, the greater the
  smoothing. In the context of SPGD, we will denote this parameter by
  $w$. This parameter can be considered independent from the rest of
  parameters.
\item The \textbf{Number of Pyramid Layers (NPL)} ($l$) which,
  depending on the slices content, can increase the length of the
  motion vectors. When the slices are very close in distance, $l=1$
  should work fine. However, if the slices are far, $l$ should be
  larger. Therefore, there is a dependency between $l$ and $\tau$
  (which controls the length of the Gaussian kernel).
\item The \textbf{Standard Deviation of the Gaussian that is used to
    smooth the cube of voxels used in the Polynomial Expansion
    coefficients (SDGPE)} ($s$) that determines the maximum spatial
  frequency that will be captured by the polynomial expansion and
  therefore, the minimum size of the recognized structures by the OF
  estimator. This parameter should not affect to the selection of the
  rest of them.
\end{enumerate}
These parameters will be determined using a similar procedure to the
used to find Eq.~\ref{eq:tau_VS_eta_empirical_EOS}. We will analyze how the corresponding SPGD parameter affects to $\hat{\eta}$.

The operation of the OF estimator also depends on other parameters
that can have fixed values and that, apart from their impact on the
computational requirements, are not susceptible to optimization (the
larger they are, the better the results should be, but the longer the
computation time is required):
\begin{enumerate}
\item The \textbf{Pyramid Scale (PS)} that determines the relative
  shape size between pyramid levels. Assuming square layers, when
  $\mathbf{PS}=0.5$, the side of the $i$-th layer is equal to half the
  side of the $i-1$-th layer.
\item The \textbf{Number Of Iterations (NOI)} that affects to the
  accuracy of the found OF fields. The higher this parameter is, the
  better the precision obtained, but the longer the computation time
  (which depends on $l$). By default, we use $\mathbf{NOI}=3$.
\item The \textbf{Size of the pixel Neighborhood used to find
    Polynomial Expansion (SNPE)} that affects on the impact of $s$ in
  the same way that $k$ influences on the frequency response of the
  Gaussian filter. SNPE should be large enought to provide a good
  stop-band attenuation, but keep in mind that SNPE can also affect to
  the cut-off frequency. We set a fixed $\mathbf{SNPE}=4$.
\end{enumerate}

{\color{red} \hrule Voy por aquí \hrule}

\begin{figure}
  \centering
  \resizebox{1.0\textwidth}{!}{
    \renewcommand{\arraystretch}{0.0} % Adjust row spacing in the table
    \setlength{\tabcolsep}{0ex}      % Adjust column spacing in the table    
    \begin{tabular}{cc}
      \href{https://nbviewer.org/github/vicente-gonzalez-ruiz/denoising/blob/main/figs/gaussian_denoising.ipynb\#GF_0MMPG_barb}{\includegraphics{GF_0MMPG_barb}} & \href{https://nbviewer.org/github/vicente-gonzalez-ruiz/denoising/blob/main/figs/OF_gaussian_denoising.ipynb\#SPGD_0MMPG_barb}{\includegraphics{SPGD_0MMPG_barb}} \\
      \href{https://nbviewer.org/github/vicente-gonzalez-ruiz/denoising/blob/main/figs/OF_gaussian_denoising.ipynb\#SPGD_SFRC_0MMPG_barb}{\includegraphics{SPGD_SFRC_0MMPG_barb}} & \href{https://nbviewer.org/github/vicente-gonzalez-ruiz/denoising/blob/main/figs/OF_gaussian_denoising.ipynb\#SPGD_PCC_0MMPG_barb}{\includegraphics{SPGD_PCC_0MMPG_barb}}
    \end{tabular}
  }
  \caption{Effect of zero-mean MPG noise in an image and how SPGD)
    can be used to reduce it. On the top-left, the best
    denoised version generated by GD. On the top-right, the best denoised version using
    SPGD. On the bottom-left it is shown the performance
    of SPGD for different levels of noise. On the bottom-right, the SFRC curves
    of the denoised image for different filter lengths. As can be seen
    in the bottom-left subfigure, for the noise level $(\sigma=30,
    \gamma=0.15)$, the optimal $\tau^*=7.0$. In the bottom-right
    subfigure it can be seen that for this noise level, the optimal
    Gaussian kernel-lenght is $\tau^*/2=3.5$.
    \label{fig:SPGD_0MMPG}}
\end{figure}

A set of experiments have been conducted to figure out:
\begin{enumerate}
\item The relationship between $\tau$, $l$, and $w$.
\item Whether the self correlation in the Fourier domain of the
  denoised image can estimate optimal values for these parameters.
\end{enumerate}

Fig.~\ref{fig:SPGD_0MMPG} shows the performance of SPGD in a artificially
noised image, for different levels of noise. As can be seen, the
optimal value for the single parameter that SPGD requires (the length of
the Gaussian kernels in each dimension) depends on the noise level, an
information that is generally unknown in microscopy imaging.

%}}}
