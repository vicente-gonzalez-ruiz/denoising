\chapter{\glsentrylong{WF}}
%{{{

The Wiener filter \cite{wiener1942extrapolation} is an adaptive linear
filter designed to minimize the MSE between the denoised signal
$\tilde{\mathbf{s}}$ and the \gls{GT} $\mathbf{s}$, when this has been
corrupted by noise\footnote{A Wiener filter also considers that
  $\mathbf{x}$ can be affected, apart from the noise, by some linear
  transformation (such as a blurring), but we will ingnore this.}. For
this reason, \gls{WF} is also known by \emph{minimum \gls{MSE} filtering} and also
by \emph{least square error filtering}. When the filter is used only
for removing noise, we are refering to denoising.

To use \gls{WF}, the next conditions must be satisfied:
\begin{enumerate}
\item The noise is not correlated with the original signal, i.e., the
  noise is, for example, \gls{AWG}.
\item The mean of the noise is zero (usually true in \gls{AWG}.
\item Signal and noise are stationary random processes (their mean and
  variance remains constant) across the signal domain.
\end{enumerate}

Under these assumptions, \gls{WF} minimizes the \gls{MSE} between the
output of the filter $\tilde{\mathbf{s}}$ and $\mathbf{s}$, i.e.,
minimizes $\mathbb{E}\big((\tilde{\mathbf{x}} -
\mathbf{x})^2\big)$. Because $\mathbf{s}$ is seldom known, \gls{WF}
uses the statistics of $\hat{\mathbf{s}}$ (the noisy signal) to try to
minimize the error. Using the \gls{IDFT}, it can be demonstrated
\cite{wiener1942extrapolation} that the signal reconstruction provided
by
\begin{equation}
  \tilde{\mathbf{s}} = \text{IDFT}(\hat{\mathbf{S}}\mathbf{W}),
  \label{eq:WF}
\end{equation}
minimizes such error, where $\hat{\mathbf{S}}$ is the \gls{DFT} of
$\hat{\mathbf{s}}$, and using the \gls{PSD},
\begin{equation}
  \mathbf{W} = \frac{\text{PSD}(\hat{\mathbf{s}})}{\text{PSD}(\hat{\mathbf{s}}) + \text{PSD}(\mathbf{n})}
  \label{eq:WF_frequency_response}
\end{equation}
is the frequency response of the Wiener filter (its transfer
function), where $\mathbf{n}$ represents the noise. If the noise is
\gls{AWG}, the \gls{PSD} can be approximated by the
variance. Therefore, Eq.~\ref{eq:WF_frequency_response} becomes
\begin{equation}
  \mathbf{W}_k = \frac{\mathbb{V}(\hat{\mathbf{S}}_k)}{\mathbb{V}(\hat{\mathbf{S}}_k) + \sigma^2_{\mathbf{n}}},
  \label{eq:WF_coeffs}
\end{equation}
where $\mathbb{V}(\hat{\mathbf{S}}_k)$ is the variance of the $k$-th
Fourier coefficient of the $\text{DFT}(\hat{\mathbf{s}})$ over a
collection of $\{\hat{\mathbf{s}}\}^{(.)}$ instances, and
$\sigma^2_{\mathbf{n}}$ is (an estimation of) the variance of the
noise. For this reason, when we have only one instance of the signal,
the filter is performed by blocks. For example, the implementation
offered in Scipy
(\href{https://docs.scipy.org/doc/scipy/reference/generated/scipy.signal.wiener.html}{\texttt{scipy.signal.wiener}}),
which implements a local Wiener filter based on the use of windows,
requires two arguments:
\begin{enumerate}
\item The window size $w$, which is the side of the (usually
  square\footnote{This implementation can work with multidimensional
    signals, and the shape of the window can be any, not only square
    (2D case) or cubic (3D case).}) window of pixels
  $\hat{\mathbf{s}}_{[i]}$ (a window with side $w$
  centered\footnote{For this reason, $w$ must be odd.} at the $i$-th
  sample of the noisy signal).
\item The noise power, expressed as the average variance of the noise
  which is locally\footnote{Only in the case this parameter, the noise
    power, is not provided. If ${\sigma^2_{\mathbf{n}}}$ is known,
    this value is used for all the windows.} computed for the window
  centered at the $i$-th sample as
  \begin{equation}
    {\sigma^2_{\mathbf{n}}}=\mathbb{E}\left(\mathbb{V}(\hat{\mathbf{s}}_{[i]})\right).
  \end{equation}
  If the \gls{GT} signal were known, the noise power\footnote{Remember that
    the noise isuncorrelated with the GT signal, and therefore the
    power of the noise is the same in all the windows, on average.}
  could be estimated as
  \begin{equation}
    {\sigma^2_{\mathbf{n}}} = \mathbb{E}\big(\mathbb{V}(\hat{\mathbf{s}}-\mathbf{s})\big).
  \end{equation}
\end{enumerate}
Concretelly, the \texttt{scipy.signal.wiener} (\texttt{WF-SS} in short) implements
\begin{equation}
  \tilde{\mathbf{s}}_i = \left\{
    \begin{array}{ll}
      \mathbb{E}(\hat{\mathbf{s}}_{[i]}) + \dfrac{\mathbb{V}(\hat{\mathbf{s}}_{[i]})-\sigma^2_\mathbf{n}}{\mathbb{V}(\hat{\mathbf{s}}_{[i]})}\left(\hat{\mathbf{s}}_i-\mathbb{E}(\hat{\mathbf{s}}_{[i]})\right) & : \ \mathbb{V}(\hat{\mathbf{s}}_{[i]}) < \sigma^2_\mathbf{n} \\
      \mathbb{E}(\hat{\mathbf{s}}_{[i]}) & : \ \text{otherwise.}
    \end{array} \right.
\end{equation}
In general, because $\sigma^2_{\mathbf{n}}$ is unknown, the denoising
process should be controlled using $w$ (the larger $w$, the greater
the filtering effect).

Alternatively, as Eq.~\ref{eq:WF} indicates, we can also denoise
$\hat{\mathbf{s}}$ in the frequency domain using as the transfer
function of the filter (in the Fourier domain) the
$\text{SFC}(\hat{\mathbf{s}})$ \cite{verbeke2024self}. In this case,
Eq.~\ref{eq:WF_frequency_response} boils down to
\begin{equation}
  \mathbf{W} = \text{SFC}(\hat{\mathbf{s}}).
  \label{eq:WF_SFC}
\end{equation}
This version of \gls{WF} will be denoted by \texttt{WF-SFC}.

\begin{comment}
If the \gls{GT} were known, Eq.~\ref{eq:WF_SFC} becomes
\begin{equation}
  \mathbf{W}(\mathbf{x}) = \text{SFC}(\mathbf{x}).
  \label{eq:WF_SFC*}
\end{equation}
We will refer to this filter as ``Wiener-SFC''.
\end{comment}

%}}}
