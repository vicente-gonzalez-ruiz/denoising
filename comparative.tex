\chapter{Final comparative}
%{{{

The experiments carried out have two main objectives:
\begin{enumerate}
\item Check the denoising performance of the tested algorithms. Since
  we have the clean version of the noisy signal, it is possible to
  establish an objective comparison between them.
\item Check whether the presence of artificial noise in the clean
  signals and their denoising generate results that are correlated
  with the previous. If the answer is ``yes'', this basically means
  that we can trust in the conclusions obtained from those experiments
  where noise is artificially incorporated.
\end{enumerate}

\begin{table}
  \centering
  \begin{tabular}{r|ccc}
    ~~ & Confocal\_FISH & TwoPhoton\_MICE & Confocal\_MICE \\
    \hline
    GF-GT & & & \\
    GF-SFC & & \\
    WF-G & & \\
    WF & & & \\
  \end{tabular}  
  \caption{PCC values obtained by different denoising algorithms
    applied to (non-artificially) noisy images for which a clean version
    is available.}
\end{table}

\begin{enumerate}
\item GF-GT: Gaussian Filtering using the Ground Truth (clean image)
  to determine, through brute-force search, the optimal standard
  deviation of the filter.
\item GF-SFC: Gaussian Filtering using the Self Fourier Correlation to
  estimate the PSD of the noise.
\end{enumerate}

%}}}
