\chapter{Distortion metrics}
In microscopy image denoising, overall distortion metrics are used to
quantify the difference between a original (``clean'' or \gls{GT})
signal $\mathbf{s}$ and its denoised version $\tilde{\mathbf{s}}$,
both with $N$ samples. These metrics help assess how well the
denoising algorithm preserved important signal features while removing
noise. Notice however, that in practice the \gls{GT} is rarely
available.

\section{\glsentryfull{MSE} and \glsentryfull{RMSE}}
%{{{

The \acrshort{MSE} is a distortion metric that quantifies the mean square
difference between the values of corresponding samples in two digital
signals, usually with the same shape. The \gls{MSE} is simple and fast to
compute, but is very sensitive to shifts or misalignments. We define
\begin{equation}
  \text{MSE}(\mathbf{s},\tilde{\mathbf{s}}) = \frac{1}{N}\sum_i(\mathbf{s}_i - \tilde{\mathbf{s}}_i)^2.
  \label{eq:MSE}
\end{equation}
Notice that the \gls{MSE} is always positive, and the lower
\gls{MSE} values, the bettet.

In the case of the \acrshort{RMSE}, we have that
\begin{equation}
  \text{RMSE}(\mathbf{s},\tilde{\mathbf{s}}) = \sqrt{\text{MSE}(\mathbf{s},\tilde{\mathbf{s}})}.
  \label{eq:RMSE}
\end{equation}
Notice that the \gls{RMSE} expresses the distortion in the same units as the
input samples.

%}}}

\section{\glsentryfull{SNR}}
%{{{

The \acrshort{SNR} compares the energy of a desired signal (\gls{GT})
$\mathbf{s}$ to the energy of \emph{background} noise
$\mathbf{s}-\tilde{\mathbf{s}}$. It is defined as the ratio of average signal
power
(\href{https://en.wikipedia.org/wiki/Expected_value}{expectation} of
the signal) to average noise power (expectation of the
noise). Therefore,

\begin{equation}
  \text{SNR}(\mathbf{x},\mathbf{y}) = \frac{\mathbb{E}(\mathbf{x})}{\mathbb{E}(\mathbf{x} - \mathbf{y})}.
  \label{eq:formal_SNR}
\end{equation}
%where, in general, 
%\begin{equation}
%  \mathbb{E}(\mathbf{x}) = \sum_{i}\mathbf{x}_iP(\mathbf{x}_i).
%  \label{eq:expectation}
%\end{equation}
Notice that if the noise has zero mean, the expectation
of the noise is also the variance of the noise.

This is the formal definition. However, because commonly in
practice the probabilities of the samples are unknown, we estimate the
expectation as a simple mean over the available data. In this case,
we estimate
\begin{equation}
  \text{SNR}(\mathbf{x},\mathbf{y}) = \frac{\sum_{i=1}^J\mathbf{x}_i^2}{\sum_{i=1}^J(\mathbf{x}_i - \mathbf{y}_i)^2}.
  \label{eq:estimated_SNR}
\end{equation}

Finally, it is common to express the power (energy) in decibels (dB).
\begin{equation}
  \text{SNR}_{\text{dB}}(\mathbf{x},\mathbf{y}) = 10\log_{10}\text{SNR}.
  \label{eq:estimated_SNR_in_dBs}
\end{equation}

A ratio higher than 1:1 (greater than 0 dB) indicates more signal than
noise. Therefore, a high SNR implies better quality (less noise).

%}}}

\section{\glsentryfull{PSNR}}
%{{{

The \acrshort{PSNR} measures the ratio between the maximum possible signal power
value ($\text{Peak}$) and the noise power, and it is defined as
\begin{equation}
  \text{PSNR} = \frac{\text{Peak}^2}{\text{MSE}}
  \label{eq:PSNR}
\end{equation}
For example, for 8 bits/pixel images, $\text{Peak}=2^8-1=255$.

%}}}

\section{\glsentryfull{SSIM} and \glsentryfull{MS-SSIM}}
%{{{

The \acrshort{SSIM} \cite{wang2004image} try to model the human
perception of the differences between two images (or volumes)
$\mathbf{x}$ and $\mathbf{y}$. The metric returns a real number
between $[-1, 1]$, $-1$ representing the perfect dis-similarity case,
$0$ no similarity, and $1$ perfect similarity. The \gls{SSIM} index is
determined (splitting the input images into $M$ non-overlapping
blocks) with
\begin{equation}
  \text{SSIM}(\mathbf{x}, \mathbf{y}) = \frac{1}{J} \sum_{j=1}^J \frac{(2\overline{\mathbf{x}}_j \overline{\mathbf{y}}_j + c_1)(2\sigma_{\mathbf{x}_j \mathbf{y}_j} + c_2)}{(\overline{\mathbf{x}_j^2} + \overline{\mathbf{y}_j^2} + c_1)(\sigma^2_{\mathbf{x}_j} + \sigma^2_{\mathbf{y}_j} + c_2)},
\end{equation}
where $\overline{\mathbf x}_j$ is the mean of the $j$-th block of
$\mathbf{x}$, $\sigma^2_{\mathbf{x}_j}$ is its variance (equivalently
for $\mathbf{y}$), $\sigma_{\mathbf{x}_j\mathbf{y}_j}$ is the
covariance (see Section \ref{sec:covariance}) of both blocks,
$c_1=(k_1L)^2$, $c_2=(k_2L)^2$ are two variables used to stabilize the
division with weak denominator, $L$ is the dynamic range of the
samples, $k_1=0.01$, and $k_2=0.03$, and where the default
size\footnote{See
  \href{https://scikit-image.org/docs/stable/api/skimage.metrics.html\#skimage.metrics.structural_similarity}{\texttt{skimage.metrics.structural\_similarity}}.}
of the local blocks is $7^D$, where $D$ is the number of signal
dimensions. When evaulating, \gls{SSIM} values below $0$ do not make
sense.

\acrshort{MS-SSIM} \cite{wang2003multiscale} is an extension of SSIM computed at
multiple image scales. It is more suitable to capture structural
similarities across different levels of detail, which can be useful in
microscopy where both fine and coarse features matter. It is defned as
\begin{equation}
  \text{MS-SSIM}(\mathbf{x}, \hat{\mathbf{y}}) = \prod_{j=1}^{J} \left[ \frac{(2 \overline{\mathbf{x}}_j \overline{\mathbf{y}}_j + c_1)(2\sigma_{\mathbf{x}_j \mathbf{y}_j} + c_2)}{(\overline{\mathbf{x}_j^2} + \overline{\mathbf{y}_j^2} + c_1)(\sigma^2_{\mathbf{x}_j} + \sigma^2_{\mathbf{y}_j} + c_2)} \right]^{\alpha_j} \left[ \frac{\sigma_j(\mathbf{x}, \hat{\mathbf{y}})}{\sigma_j(\mathbf{x}) + \sigma_j(\hat{\mathbf{y}}) + c_3} \right]^{\beta_j}
\end{equation}
where $\mathbf{x}$, $\mathbf{y}$ are the original and approximated
images/volumes, respectively, $\overline{\mathbf{x}}_j$,
$\overline{\mathbf{y}}_j$ are the local means of the $j\text{-th}$
block of $\mathbf{x}$, and $\hat{\mathbf{y}}$, respectively,
$\sigma_{\mathbf{x}_j \mathbf{y}_j}$ is the covariance between the
blocks of $\mathbf{x}_j$ and $\hat{\mathbf{y}}_j$,
$\sigma_{\mathbf{x}_j}, \sigma_{\mathbf{y}_j}$ are the standard
deviations of the blocks of $\mathbf{x}_j$ and $\mathbf{y}_j$,
respectively, and $\alpha_j, \beta_j$ are the weights that control the
importance of each term at the $j\text{-th}$ scale/block.

%}}}

\section{\glsentryfull{PCC}}
%{{{

The \acrshort{PCC} is
\href{https://en.wikipedia.org/wiki/Pearson_correlation_coefficient}{given
  by}
\begin{equation}
  \text{PPC}(\mathbf{x}, \mathbf{y}) = \frac{\sum_j(\mathbf{x}_j - \overline{\mathbf{x}})(\mathbf{y}_j - \overline{\mathbf{y}})}{\sqrt{\sum_j (\mathbf{x}_j - \overline{\mathbf{x}})^2 \sum_j (\mathbf{y}_j - \overline{\mathbf{y}})^2}},
  \label{eq:PCC}
\end{equation}
and measures the linear correlation between two signals $\mathbf{x}$
and $\mathbf{y}$.  As happen with the SSIM, the output is in
$[-1, -1]$, meaning $-1$ a perfect negative linear relationship
between the input tensors, $0$ no linear relation, and $1$ a perfect
coincidence.

Notice that the \gls{PCC} is the same that the cross-correlation
only for the 0-displacement when the signals are normalized. Therefore,
although the computation of the \gls{CC} has a computational complexity of
$O(N^2)$, we can determine the \gls{PCC} in $O(N)$.

%}}}

\section{\glsentryfull{LPIPS}}
%{{{

\acrshort{LPIPS} \cite{zhang2018unreasonable} is a deep learning-based
metric that uses a pretrained neural network (e.g., AlexNet and VGG)
to assess perceptual similarity, providing a better correlation with
the human perfection of the distortion. The idea is to use the feature
maps (intermediate activations) from various layers of a pre-trained
\gls{CNN} to extract features from the input images. These feature maps
capture important perceptual information, such as textures, edges, and
high-level structures. Then, the similarity is computed in terms of
feature differences between corresponding patches of the
images. Concretelly,
\begin{equation}
  \text{LPIPS}(\mathbf{x}, \mathbf{y}) = \sum_i=1^Lw_i||f_i(\mathbf{x}) - f_i(\mathbf{y})||_2
\end{equation}
where $w_i$ is the weight (contribution) of the layer $i$ to the
metric, $L$ is the number of layers used in the comparison,
$f_i(\mathbf{x})$ are the weights of the feature map at layer $i$, and
$||f_i(\mathbf{x}) - f_i(\mathbf{y})||_2$ measures the distance (see
Appendix \ref{sec:L2_norm}) between the features at that layer.

\gls{LPIPS} is trained using a large dataset of human judgments for image
similarity, where human observers are asked to rate the perceptual
similarity between pairs of images. The network is trained to find the
$w_i$ which minimize the difference between predicted perceptual
similarity and human ratings.

%}}}

\section{\glsentryfull{NIQE}}

To do.

% A. Mittal, R. Soundararajan, and A. C. Bovik, “Making a “completely
% blind” image quality analyzer,” IEEE Signal Processing Letters, vol. 20,
% no. 3, pp. 209–212, 2013.

\section{\glsentryfull{CS}}
The \acrshort{CS} is a metric used to measure the similarity between
two non-zero vectors in a multi-dimensional space. It quantifies how
similar two vectors are in their direction or orientation, rather than
their magnitude or length.

\begin{equation}
  \cos(\mathbf{A}, \mathbf{B}) = \frac{\mathbf{A}\cdot\mathbf{B}}{||\mathbf{A}||~||\mathbf{B}||},
\end{equation}
where $\mathbf{A}\cdot\mathbf{B}$ is the dot product of vectors
$\mathbf{A}$ and $\mathbf{B}$, and $||A||$ is the magnitude (Euclidean
norm) of vector $\mathbf{A}$.
