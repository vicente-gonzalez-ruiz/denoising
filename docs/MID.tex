\chapter{MID (Multimodal Iterative Denoising)}

% MID: A Self-supervised Multimodal Iterative Denoising Framework C Nie, T Deng, Z Liu, H Wang - arXiv preprint arXiv:2511.00997, 2025

Learns noise characteristics directly from noisy inputs, applying a
Taylor expansion to handle non-linear contamination.

Operates in a self-supervised manner, learning to denoise using only
noisy data as input. This is achieved by further corrupting
already noisy data and learning to reverse these incremental
additions, which eliminates the need for pristine ground-
truth samples. The framework employs an iterative process
to subtract estimated noise, gradually recovering the clean
signal while preserving fine details.

The application of a
Taylor expansion allows it to approximate complex non-linear
noise contamination as a sequence of linear perturbations,
enabling effective iterative removal.


MID trains a system to recognize and remove noise by
learning from data, denoted by s, that is progressively cor-
rupted. The framework, depicted in Fig. 2, involves two
main phases: a training phase where noise characteristics are
learned, and a denoising phase where the trained networks are
used to remove noise from new, unseen data

