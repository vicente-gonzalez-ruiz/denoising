Non-Local Means (NLM)

Key idea:
Instead of averaging only nearby pixels (like Gaussian blur) or applying frequency filtering, NLM denoises each pixel by taking a weighted average of similar patches across the whole image.

How it works:

For each pixel 
i
i, take a small patch 
Pi
P
i
	​

 around it.

Search for other patches 
Pj
P
j
	​

 in a large search window.

Compute similarity between 
Pi
P
i
	​

 and 
Pj
P
j
	​

 (typically using a Gaussian-weighted Euclidean distance).

Assign higher weights to more similar patches.

The denoised value of pixel 
i
i is a weighted average of the central pixels of all candidate patches.

Properties:

Preserves textures and repetitive structures well (since it leverages patch self-similarity).

Computationally heavy (many patch comparisons).

Sensitive to parameter choices (patch size, search window, filtering parameter).

BM3D (Block-Matching and 3D Filtering)

Key idea:
BM3D improves upon NLM by grouping similar patches and performing collaborative filtering in a transform domain (like wavelet/DCT).

How it works:

Block Matching: For a reference patch, find a group of similar patches across the image.

Stacking: Arrange them into a 3D block (a stack of similar patches).

Collaborative Filtering: Apply a 3D transform (2D spatial transform + 1D transform along the stack dimension), shrink small coefficients (thresholding or Wiener filtering), and invert the transform.

Aggregation: Place filtered patches back into the image and average overlapping contributions.

Properties:

More advanced than NLM, because it not only averages but also filters in a sparse transform domain.

Produces state-of-the-art denoising results among classical methods.

Computationally intensive, but usually faster than brute-force NLM with large search windows.

Key Differences: NLM vs. BM3D
Aspect	NLM	BM3D
Similarity search	Finds similar patches, averages them	Finds similar patches, groups them
Filtering	Weighted averaging in image domain	Transform-domain shrinkage (sparsity-based)
Denoising strength	Good, but limited	Superior, one of the best classical methods
Texture preservation	Decent, but may blur fine details	Excellent (keeps sharp edges & textures)
Computation	Heavy (all-pair comparisons)	Also heavy, but more efficient due to grouping
Output quality	Can leave residual noise / oversmooth	Produces cleaner, sharper results