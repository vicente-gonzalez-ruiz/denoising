\chapter{Intro}
%{{{

%\begin{abstract}
%{{{

  Microscopy of biological specimens is used in biotechnology to
  visualize small specimens (including their internal structures) that
  cannot be seen with the naked eye. For this, some kind of
  interaction between the specimens and a source of energy must exist
  (the specimen must be irradiated), an action that usually degrades
  the specimen and for this reason, the radiation is minimized
  generating low SNR images. In this report, we describe, evaluate,
  and compare several image denoising algorithms commonly used in
  microscopy.

%}}}
%\end{abstract}
\section*{Definitions and notation}
%{{{

\begin{tabular}{ll}
  $x$ & A scalar value (e.g., a value of a pixel of a grayscale image) \\
  $s(t)$ & A (continuous) signal as a function of time \\
  $s[n]$ & A discrete signal (only) defined at instants of time $tn, n\in\mathcal{Z}, t>0$ \\
  $\mathbf{s}$ & A digital (discrete and finite) signal (e.g., an image) \\
  $\mathbf{s}_{i}$ & The $i$-th element of $\mathbf{s}=\{\mathbf{s}_{i}\}_{i=0}^{N-1}=\{\mathbf{s}_{i}\}$ \\
  %$A[b]$ & The $b$-th element of the sampled version of $A(b)$ \\
  $\{i\}$ & The set $i$ \\
  $\mathbf{s}_{\{i\}}$ & The elements of $\mathbf{s}$ with indices $\{i\}$ \\
  $\mathbf{s}_{[i]}$ & A window of samples of $\mathbf{s}$ centered at the $i$-th sample \\
  $\mathbf{s}_{\href{https://numpy.org/doc/stable/user/basics.indexing.html#slicing-and-striding}{y,:}}$ & The $y$-th row of the image $\mathbf{s}$ \\
  $\mathbf{s}_{:,x}$ & The $x$-th column of the image $\mathbf{s}$ \\
  $\mathbf{s}_{y,x}$ & The pixel $(y,x)$ of the image $\mathbf{s}$ \\
  %$\mathbf{x}^{(i)}$ & The $i$-th real-noisy instance of the signal $\mathbf{x}$ \\
  %$\mathbf{s}^{()}$ & An instance of $\mathbf{s}$, possibly noisy \\
  $\mathbf{s}^{(i)}$ & The $i$-th instance of the signal $\mathbf{s}$ \\
  $\tilde{\mathbf{s}}^{(I)}$ & Approximation to $\mathbf{s}$ using $I$ instances \\ 
  $\overline{\mathbf{s}}$ & A mean of the samples of $\mathbf{s}$ \\ 
  $\href{https://docs.python.org/3/library/functions.html#len}{\text{len}}(\mathbf{s})$ & $=\mathbf{s}.\href{https://numpy.org/doc/stable/reference/generated/numpy.ndarray.size.html}{\mathsf{size}}$ Number of elements in $\mathbf{s}$ \\
  $\href{https://numpy.org/doc/stable/reference/generated/numpy.shape.html}{\text{shape}}(\mathbf{s})$ & ($=\mathbf{s}.{\mathsf{shape}}$) Shape of $\mathbf{s}$ \\
  $\text{rank}(\mathbf{s})$ & ($=\mathbf{s}.\mathsf{rank}=\text{len}(\mathbf{s}.\mathsf{shape})$) Dimensionality of $\mathbf{s}$ \\
  $\mathsf{\href{https://docs.python.org/3/library/functions.html\#func-range}{range}}(s)$ & $=\{0, 1, \cdots, s-1\}$ \\
  $\mathsf{\href{https://numpy.org/doc/stable/reference/generated/numpy.zeros_like.html}{zeros\_like}}(\mathbf{s})$ & $=\{0\}_{i=0}^{\mathbf{s}.\mathsf{size}-1}$ \\
  % $|\mathbf{X}_i|$ & The absolute value of $\mathbf{X}_i$ \\
  $\alpha\mathbf{s}$ & $=\{\alpha\mathbf{s}_i\}$ (scalar multiplication) \\
  $\mathbf{x}+\mathbf{y}$ & $=\{\mathbf{x}_i + \mathbf{y}_i\}$ (Hadamard addition) \\ 
  $\mathbf{x}\mathbf{y}$ & $=\{\mathbf{x}_i\mathbf{y}_i\}$ (Hadamard product) \\ 
  $\mathcal{N}$ & The normal distribution \\ 
  $\mathcal{P}$ & The Poisson distribution \\
  $\mathbf{x}\sim\mathcal{N}$ & The elements of $\mathbf{x}$ follows a normal distribution \\
  $\mathbf{x}_{\mathcal{N}}$ & The same as $\mathbf{x}\sim\mathcal{N}$ \\
  $\Pr(\mathbf{x}=a)$ & Probability that a $\mathbf{x}_i$ takes the value $a$ \\
  $\Pr(\mathbf{x}=a, \mathbf{y}=b)$ & $\Pr(\mathbf{x}=a)$ and $\Pr(\mathbf{y}=b)$ (joint probability)  \\
  $\mathrm{Su}(\mathbf{x})$ & $=\{x\in\mathbb{R}|\Pr(\mathbf{x}=x)>0\}$ (support of $\mathbf{x}$)\\
  $\Pr(A|B)$ & Conditional probability of $A$ given $B$ \\
  $\mathbb{E}(\mathbf{s})$ & Expectation of $\mathbf{s}$ \\
  $\mathbb{V}(\mathbf{s})$ & Variance of $\mathbf{s}$ \\
  $||\mathbf{s}||_2$ & $L_2$ norm of $\mathbf{s}$ \\
  $f_s$ & Sampling frequency \\
  $\mathcal{F}$ & The (forward) Fourier transform ($\mathcal{F}(\mathbf{s})=\mathbf{S}$) \\
  $\mathcal{F}^{-1}$ & The inverse Fourier transform ($\mathcal{F}^{-1}(\mathbf{S})=\mathbf{s}$) \\
  $\cdot^*$ & the complex conjugate of $\cdot$ \\
  $\mathbf{x}*\mathbf{y}$ & $=\mathcal{F}^{-1}(\mathcal{F}(\mathbf{x})\mathcal{F}(\mathbf{y}))=\mathcal{F}^{-1}(\mathbf{X}\mathbf{Y})$ (convolution) \\
  $A(b)$ & $A$ depends on (parameter) $b$ \\
  $A.b$ & The $b$ component of the data structure $A$ \\
  $(A)b$ & First $A$, then $b$ 
\end{tabular}

%}}}

\section{Imaging techniques}
%{{{

There are several tecniques to capture microscopy images.

In light microscopy (LM), light passes through a specimen, and this
transmitted light is then magnified by the objective and ocular lenses
to form an observable image. Fluorescence microscopy and confocal
microscopy are two LM tecniques, where the specimen(s) emit light
after have being excited by a source of light. Typical resolution: 200
nm.

In electron microscopy (EM) we use a beam of electrons that can pass
through the sample (TEM (Transmission Electron Microscopy)) or bounce
on the sample (SEM (Scanning Electron Microscopy)). X-rays and ions
can be also used (the smaller the wavelength of the bean, the higher
the resolution). In general, compared to LM, EM allows much higher
resolution (down to sub-nanometer scale).

Finally, there is a third techinique called atomic force microscopy
(AFM), in which a nano-scale mechanical probe, a scanning probe
microscopy maps the surface of a specimen.

Unfortunately, the energy that impact on the specimen also degrades
it, specially in the case of biological specimens. To minimize this
degradation, the radiation doses are minimized, generating low
signal-to-noise ratio images.

%}}}

\section{Denoising in microscopy}
%{{{

%We consider denoising techniques where there is only one noisy
%instance or a few instances, usually in the latter case with slightly
%different versions of the clean signal. Averaging as such is therefore
%excluded.

Microscopy images are characterized by a low signal-to-noise ratio
(SNR), requiring the use of denoising algorithms to facilitate
subsequent analysis. A common challenge is the limited availability of
only a single noisy realization of these images, which complicates
both the evaluation of denoising algorithms (due to the absence of a
ground truth for comparative assessment) and the selection of
appropriate denoising parameters. An optimal denoiser should
effectively attenuate noise while concurrently preserving crucial
image features, including edges, textures, and underlying biological
structures.

In all the denoising techniques described here we will supose that
both, the images and the noise can be modeled as \emph{stationary
  random variables}. A stationary random variable can be described as
a stochastic process whose statistical properties do not change. This
means that the probability distribution of the process at any given
set of time points is the same as the distribution at those same time
points shifted by any constant amount in time or space.

%}}}

%}}}
