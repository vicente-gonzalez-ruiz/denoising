\chapter{Introduction}

Microscopy of biological specimens is used in biotechnology to
visualize small specimens (including their internal structures) that
cannot be seen with the naked eye. For this, some kind of interaction
between the specimens and a source of energy must exist (the specimen
must be irradiated), an action that usually degrades the specimen and
for this reason, the radiation is minimized generating low SNR
images. In this report, we describe, evaluate, and compare several
image denoising algorithms commonly used in microscopy of biological
specimens.

\section{Imaging techniques in microscopy}
%{{{

There are several tecniques to capture microscopy images.

In \gls{LM}, light passes through the specimen, and this light is then
magnified by an objective and ocular lenses to form an observable
image. Fluorescence microscopy and confocal microscopy are two
\gls{LM} tecniques, where the specimen(s) emit light after have being
excited by a source of light. Using \gls{LM} we can typical reach a
resolution of up to 200 nm.

In \gls{EM} we use a beam of electrons that can pass through the
sample (\gls{TEM}) or bounce on the sample (\gls{SEM}). X-rays and
ions can also be used (the smaller the wavelength of the radiation,
the higher the resolution). In general, compared to \gls{LM}, \gls{EM}
allows much higher resolution (down to sub-nanometer scale).

Finally, there is a third imaging techinique called \gls{SPM}, in
which a nano-scale mechanical probe maps the surface of a
specimen. \gls{SPM} achieves sub-nanometer resolution, allowing for
the imaging of surfaces at the atomic scale. The \gls{AFM} and the
\gls{STM} are \gls{SPM} techniques.

%}}}

\section{Why denoising?}
\label{sec:why_denoising}
%{{{

%We consider denoising techniques where there is only one noisy
%instance or a few instances, usually in the latter case with slightly
%different versions of the clean signal. Averaging as such is therefore
%excluded.

In general, when we study biological specimens, the impact of the
ratiation of generated by the microscope on the specimen also degrades
it. To minimize this degradation, the radiation doses are minimized,
generating low signal-to-noise ratio images. Therefore, microscopy
images are characterized by a low \gls{SNR}, requiring the use of
denoising algorithms to facilitate subsequent analysis. A common
challenge is the limited availability of only a single noisy
realization of these images, which complicates both the evaluation of
denoising algorithms (due to the absence of a ground truth for
comparative assessment) and the selection of appropriate denoising
parameters. An optimal denoiser should effectively attenuate noise
while concurrently preserving crucial image features, including edges,
textures, and underlying biological structures.

In all the denoising techniques described here we will supose that
both, the images and the noise can be modeled as \gls{SRV}s. A
\gls{SRV} can be described as a stochastic process whose statistical
properties do not change. This means, for example, that the
probability distribution of a signal at any given set of time points
is the same as the distribution at those same time samples shifted by
any constant amount in time or space.

%}}}

