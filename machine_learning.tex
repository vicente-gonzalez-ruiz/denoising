\chapter{Machine learning}

\section{Multivariate data}
%{{{

Multivariate data refers to datasets that contain more than two
variables for each observation. These variables represent different
characteristics or measurements related to the observed phenomenon. In
simpler terms, if you are collecting information about something and
recording three or more different aspects for each item, you are
dealing with multivariate data. Example: A study on students: You
might collect data on their age, test scores in math, test scores in
science, attendance rate, and extracurricular activities. Each student
is an observation, and the five pieces of information are the multiple
variables.

%}}}

\section{Training of a regression model for denoising} % https://arxiv.org/pdf/1803.04189
A regression model, e.g., a \gls{CNN} can be trained using a number of pairs $(\hat{{\mathbf x}}_i, {\mathbf y}_i)$ of corrupted inputs $\hat{{\mathbf x}}_i$ and clean targets ${\mathbf y}_i$, and minimizing the empirical risk
\begin{equation}
  \underset{\theta}{\operatorname{arg\,min}} \, \sum_i L \big(f_\theta(\hat{\mathbf x}_i), {\mathbf y}_i\big)
\end{equation}
where $f_\theta$ is a parametric family of mappings (e.g., CNNs) under the loss function $L$.

