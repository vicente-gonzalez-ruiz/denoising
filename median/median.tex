\chapter{Median filtering}

Median filtering is a non-linear digital filtering technique used to
remove noise from signals. It's particularly effective at removing
``salt and pepper'' noise, which appears as random black and white
pixels in an image. 

The median filter works by moving a "window" or "kernel" (typically a
small square) over each pixel of the image. For each position of the
window, the filter gathers the brightness values of all the pixels
within that window, sorts them, and replaces the central pixel's value
with the median value from that sorted list.

\section{Generalized median filter}
Extends the median filter to vectors of data (for example, RGB
pixels). For each pixel, it considers the color vectors of its
neighbors. It then calculates the distance between each vector and all
the others. The vector with the smallest total distance to all other
vectors is chosen as the ``generalized median''.
