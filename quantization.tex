\chapter{Quantization of discrete signals}

Quantization is a fundamental concept in digital signal processing
that involves mapping a continuous or high-resolution discrete signal
to a finite set of levels. It is a key step in converting analog
signals to digital form. Quantization is a non-linear operation that
introduces quantization error.

Notice that this concept implies as input a discretized signal in the
signal domain (in general, \emph{analog} signal samples) and the
output is a digital signal (where the signal samples are discretized
in amplitude). However, quantization can be also used to re-encode
already quantized signals (re-quantization), a process that is
involved in lossy data compression. A similar idea is used when we
quantize the parameters of machine learning model.

\section{Definition}
In general, quantization is a mathematical injective projection
between the elements of two sets $A$ and $B$, where $|A|>|B|$, and
obviously, this is an irreversible action.

In the context of discrete signals, quantization of signals the process
of approximating a continuous range of values (or a large discrete
set) with a limited number of levels.

\section{The quantization error}

The quantization error is the difference between the
discrete-but-still-analog input signal and the digital output signal
\begin{equation}
  e_n = s_n - \hat{s}_n
\end{equation}

Under a high enough number of quantization levels, quantization error
is typically modeled as a uniform random variable.
