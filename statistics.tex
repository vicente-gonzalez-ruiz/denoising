\chapter{Statistics for digital data}

\section{Random variable}
A random variable is a function
\begin{equation}
  X:\Omega\rightarrow \mathcal{X}
\end{equation}
that maps outcomes $\omega\in\Omega$ (the sample space) to numerical values in some set $\mathcal{X}$ (usually $\mathbb{R}$, or a countable/continuos subset). For example, if we toss a coin, $X$ is a random variable that numerically represents the coin flip, $\omega=\{\text{Heads},\text{Tails}\}$, by definition $X(\omega)=1$ if $\omega=\text{Heads}$, and $X(\omega)=0$ if $\omega=\text{Tails}$.

Notice that a randon variable is not random itself (it's a deterministic function of the underlying outcome). The ``randomness'' comes from the fact that the outcome $\omega$ is random (governed by $P$).

\section{Probability mass function}
The \gls{PMF} of a random variable $X$, defined as
\begin{equation}
  p(x) = \Pr(X=x), \quad x\in\mathcal{x},
\end{equation}
describes the probability with which a random variable $X$ takes values in a countable set $\mathcal{X}$, where
\begin{equation*}
  p(x)\ge 0,\quad\sum_{x\in\mathcal{X}}p(x)=1.
\end{equation*}

% We will assume that discrete signals are stationary (discrete) random processes

Discrete data only can take countable values (like 0, 1, 2, ...).

Random variables

A digital signal is the result of discretizing in amplitude a discrete
signal. Therefore, a sample of a digital signal can be represented
using a finite number of bits.

By physical requirements, digital signals are finite in length, and therefore
$\text{Sup}(\mathbf{x})$ is a finite set.

\section{Discrete signals and discrete random variables}
\label{sec:DSRV}
%{{{

Mathematically, we can model discrete signals as \glspl{SRV}. A
\gls{SRV} can be described as the output of a stochastic process whose
statistical properties do not change with time or space. This means,
basically, that the probability distribution of a signal at any given
set of time/space points is the same as the distribution at those same
samples shifted by any constant amount in time or space.

A digital signal is the results of discretizing a discrete signal in
amplitude. By physical reasons, a digital signal is finite in length
(the number of signal samples is a known number) and in amplitude
(each sample requires a finite number of bits to be
represented). Therefore, digital signals are discrete and finite.

A random variable is a mathematical formalization of a quantity or
object which depends on random events. A \gls{DRV} is a random
variable that only can take a finite number of different values. When
more than one value (a vector) is generated in one of these events, we
can establish a similar connection between multicomponent signals and
random vectors. Statistically, random variables can be described
throught the corresponding probability
distribution. In the case of (discrete) random vectors, a mean for each
component, a covariance matrix, and a multivariate distribution are
required.

In this document, monocomponent (digital) signals (and therefore
\glspl{DRV}) will be denoted with lower-case bold-faced symbols,
such as $\mathbf{x}$. Multicomponent (digital) signals (random
vectors) will be denoted as $\overrightarrow{\mathbf{x}}$.

%}}}

\section{Expectation of \glspl{DRV}}
The expected value (or expectation, mathematical expectation,
expectancy) of a \gls{DRV} $\mathbf{x}$, denoted as
$\mathbb{E}(\mathbf{x})$, is a theoretical concept representing the
long-run average value of the variable if the experiment were to be
repeated an infinite number of times. It is defined as the weighted
average of all possible values the random variable can take, where the
weights are the probabilities of those values occurring.

For a digital signal $\mathbf{x}$, it is defined as
\begin{equation}
  \mathbb{E}_{\mathbf{x}}(\mathbf{x})=\mathbb{E}(\mathbf{x})=\sum_i\mathbf{x}_i\Pr(\mathbf{x}=\mathbf{x}_i).
  \label{eq:expectation}
\end{equation}

The subscripts in expectations (like $\mathbb{E}(X,Y)_{(X,Y)}$) are
there to indicate with respect to which probability distribution the
expectation is being taken.

The expectation is a linear operator and satisfies
\begin{equation}
  \mathbb{E}(\mathbf{x}+\mathbf{y}) = \mathbb{E}(\mathbf{x}) + \mathbb{E}(\mathbf{y}),
\end{equation}
and
\begin{equation}
  \mathbb{E}(c\mathbf{x}) = c\mathbb{E}(\mathbf{x}),
\end{equation}
where $c$ is a scalar constant.

Notice that using the expectation operator, the power of a digital
signal (see Eq.~\ref{eq:power_discrete_signal}) can be found as
\begin{equation}
  P(\mathbf{x}) = \mathbb{E}(|\mathbf{x}|^2).
  \label{eq:power_as_expectation}
\end{equation}

\section{Expectation of the function of a discrete random variable}
\begin{equation}
  \mathbb{E}[f(\mathbf{x})]=\sum_if(\mathbf{x}_i)\Pr(\mathbf{x}=\mathbf{x}_i).
  \label{eq:expectation_of_function}
\end{equation}

\section*{Example:}
Suppose we want to compute the expected winnings of a simple betting game tossing a coin, where if Heads, you win \$2, but if Tails, you lose \$1. We define the payoff as a function of $X$:

\begin{equation*}
  f(X)=2X-1
\end{equation*}

Therefore, if $X=1$ (Heads): $f(1)=2(1)-1=1$ (dollar net gain), but if $X=0$ (Tails): $f(0)=2(0)-1=-1$ (dollar loss). The expectation of the payoff is

\begin{equation*}
\mathbb{E}[f(X)]=f(1)\Pr(X=1)+f(0)\Pr(X=0) = (1)(12)+(-1)(12)=0.
\end{equation*}

So your expected profit in this game is zero.

\section{Marginal of a random variable}
The marginal of a variable is its probability distribution, ignoring
the other variables. For example, imagine we survey a neighborhood and
record the type of pet (cat or dog) and its color (brown or
black). The marginal of pet type considers just the number of cat and
dogs, not their color.

\section{Conditional distribution}

The conditional probability mass function is
$$ p(y \mid x) = \frac{p(x,y)}{p(x)} \quad \text{if } p(x) > 0, $$
where $p(x) = \sum_{y} p(x,y)$ is the marginal of $X$.

The conditional expectation of $h(x,Y)$ given $X=x$ is then
$$ E_{Y \mid X=x}[h(x,Y)] = \sum_{y} h(x,y) p(y \mid x). $$

\section{Low of total expectation}
The expectation with respect to the joint distribution is
\begin{equation}
  \mathbb{E}_{(X,Y)}[h(X,Y)]=\mathbb{E}_{X}[\mathbb{E}_{Y\mid X}[h(X,Y)]].
\end{equation}

\section*{Example: Coin + Die}

Toss a fair coin and roll a fair die simultaneously.

\subsection*{Random Variables}
\[
C = 
\begin{cases}
1 & \text{if Heads} \\
0 & \text{if Tails}
\end{cases}, 
\qquad
D \in \{1,2,3,4,5,6\}.
\]

The coin and die are independent, so the joint distribution is
\[
p(c,d) = p(c)\,p(d) = \frac{1}{2} \cdot \frac{1}{6} = \frac{1}{12}.
\]

\subsection*{Function of Interest}
Suppose the payoff is
\[
h(C,D) = C \cdot D,
\]
i.e., you get the die roll if the coin shows Heads, and 0 if Tails.

\subsection*{Expectation w.r.t. the joint distribution}
\[
\mathbb{E}_{(C,D)}[h(C,D)] 
= \sum_{c=0}^{1} \sum_{d=1}^{6} h(c,d) \, p(c,d).
\]

Plugging in:
\[
\mathbb{E}_{(C,D)}[h(C,D)] 
= \sum_{d=1}^{6} (1 \cdot d) \cdot \frac{1}{12} + \sum_{d=1}^{6} (0 \cdot d) \cdot \frac{1}{12}
= \frac{1}{12} (1+2+3+4+5+6) = \frac{21}{12} = 1.75.
\]

\subsection*{Law of Total Expectation}
We can also write
\[
\mathbb{E}_{(C,D)}[h(C,D)] = \mathbb{E}_C \Big[ \mathbb{E}_{D|C}[h(C,D)] \Big].
\]

Compute the inner conditional expectation:
\[
\mathbb{E}_{D|C=1}[h(1,D)] = \mathbb{E}_D[D] = 3.5, \quad 
\mathbb{E}_{D|C=0}[h(0,D)] = \mathbb{E}_D[0] = 0.
\]

Then average over $C$:
\[
\mathbb{E}_C \Big[ \mathbb{E}_{D|C}[h(C,D)] \Big] 
= 0.5 \cdot 3.5 + 0.5 \cdot 0 = 1.75.
\]

\subsection*{Conclusion}
The subscripts in $\mathbb{E}_{(\cdot)}$ clarify which distribution is used for averaging:
\begin{itemize}
    \item $\mathbb{E}_{(C,D)}$: joint distribution,
    \item $\mathbb{E}_{D|C}$: conditional distribution of $D$ given $C$,
    \item $\mathbb{E}_C$: marginal distribution of $C$.
\end{itemize}

\section{$n$-th moment of a discrete random variable}
\begin{equation}
  \mathbb{E}(\mathbf{x}^n) = \sum_i\mathbf{x}_i^n\Pr(\mathbf{x}=\mathbf{x}_i).
\end{equation}
  
\section{Mean}
\label{sec:mean}
The mean of a discrete random variable is the first moment of the  

\section{Average}
\label{sec:average}
In statistics, when referring to a random variable, the mean (also
called ``sample mean'' or ``empirical mean'') usually refers to the
expectation. However, when we deal with a digital signal (sample data)
$\mathbf{x}$, we define
\begin{equation}
  \overline{\mathbf{x}} = \frac{1}{N}\sum_{i=0}^{N-1}\mathbf{x}_i,
  \label{eq:mean}
\end{equation}
where $N$ is the number of samples in $\mathbf{x}$. We will also use
$\mu$ to denote the mean value.

Notice that Eq.~\ref{eq:mean} is equivalent to
Ed.~\ref{eq:expectation} when all the samples have the same
probability.


\section{Variance}
\label{sec:variance}
%{{{

The variance of a random variable $\mathbf{x}$, denoted by
$\mathbb{V}(\mathbf{x})$ (or simply by $\sigma^2$) is a measurement of
its dispersion. It is defined as the expected value of the squared
deviation from the mean,
%\begin{equation}
%  \operatorname{Var}(\mathbf{x}) = \mathbb{E}\left[(X - \mathbb{E}[X])^2 \right].
%\end{equation}
%\begin{equation}
%  \operatorname{Var}(\mathbf{x}) = \operatorname{Exp}\left[(X - \operatorname{Exp}[X])^2 \right].
%\end{equation}
\begin{equation}
  \mathbb{V}(\mathbf{x}) = \mathbb{E}\left((\mathbf{x} - \mathbb{E}(\mathbf{x}))^2 \right) = \mathbb{E}(\mathbf{x}^2)-\mathbb{E}(\mathbf{x})^2.
  \label{eq:variance}
\end{equation}

%}}}

\section{Power of a signal as a function of its mean and variance}
The average power of a random variable is equal to its variance plus its mean
(expectation) squared
\begin{equation}
  P(\mathbf{x}) =  \mathbb{V}(\mathbf{x}) + \mathbb{E}(\mathbf{x})^2.
  \label{eq:power_mean_variance}
\end{equation}
This is because
\begin{align*}
  \mathbb{E}(\mathbf{x}^2)
  & = \mathbb{E}(\mathbf{x}^2 - \mathbb{E}(\mathbf{x}) + \mathbb{E}(\mathbf{x})) \\
  & = \mathbb{E}((\mathbf{x} - \mathbb{E}(\mathbf{x}))^2 + 2\mathbb{E}(\mathbf{x})(\mathbf{x}-\mathbb{E}(\mathbf{x})) + \mathbb{E}(\mathbf{x})^2) \\
  & = \mathbb{E}((\mathbf{x}-\mathbb{E}(\mathbf{x}))^2) + 2\mathbb{E}(\mathbf{x})(\mathbb{E}(\mathbf{x})-\mathbb{E}(\mathbf{x})) + \mathbb{E}(\mathbf{x})^2) \\
  & = \mathbb{V}(\mathbf{x}) + 0 + \mathbb{E}(\mathbf{x})^2.
\end{align*}

\section{Covariance}
\label{sec:covariance}
%{{{

The covariance $\mathbb{C}(\mathbf{x}, \mathbf{y})$ is a measure of
how two random variables $\mathbf{x}$ and $\mathbf{y}$ change
together. In simpler terms, it tells us the direction of the linear
relationship between two variables. The covariance between two
discrete signals $\mathbf{x}$ and $\mathbf{y}$ is calculated as
%\begin{equation}
%  \text{Cov}(\textbf{x}, \textbf{y}) = \mathbb{E}[(\mathbf{x}-\overline{\mathbf{x}})(\mathbf{y}-\overline{\mathbf{y}})].
%\end{equation}
%\begin{equation}
%  \mathbb{V}(\mathbf{x}) = \mathbb{E}\left((\mathbf{x} - \mathbb{E}(\mathbf{x}))^2 \right).
%\end{equation}
\begin{equation}
  \mathbb{C}(\textbf{x}, \textbf{y}) = \mathbb{E}\big((\mathbf{x}-\mathbb{E}(\mathbf{x}))(\mathbf{y}-\mathbb{E}(\mathbf{y}))\big).
\end{equation}

Notice that (see Eq.~\ref{eq:variance})
\begin{equation}
  \mathbb{C}(\mathbf{x}, \mathbf{x}) = \mathbb{V}(\mathbf{x}).
\end{equation}

%}}}

\section{Covariance matrix}
\label{sec:covariance_matrix}
%{{{

The covariance matrix $\Sigma_{\overrightarrow{\mathbf{x}}}$ of a random vector $\overrightarrow{\mathbf{x}}=[\mathbf{x}_1,\cdots,\mathbf{x}_N]^T$, defined as,
\begin{equation}
  (\Sigma_{\overrightarrow{\mathbf{x}}})_{i,j}=\mathbb{C}(\mathbf{x}_i,\mathbf{x}_j),
\end{equation}
is a $N\times N$ matrix
\begin{equation}
\Sigma_{\overrightarrow{\mathbf{x}}} = 
\begin{pmatrix}
\mathbb{V}(\mathbf{x}_1) & \mathbb{C}(\mathbf{x}_1, \mathbf{x}_2) & \cdots & \mathbb{C}(\mathbf{x}_1, \mathbf{x}_p) \\
\mathbb{C}(\mathbf{x}_2, \mathbf{x}_1) & \mathbb{V}(\mathbf{x}_2) & \cdots & \mathbb{C}(\mathbf{x}_2, \mathbf{x}_p) \\
\vdots & \vdots & \ddots & \vdots \\
\mathbb{C}(\mathbf{x}_p, \mathbf{x}_1) & \mathbb{C}(\mathbf{x}_p, \mathbf{x}_2) & \cdots & \mathbb{V}(\mathbf{x}_p)
\end{pmatrix}
\end{equation}
that express the covariance between the random variables of a random vector.

%}}}

\section{$L_2$ norm}
\label{sec:L2_norm}
%{{{

$L_2$ norm (also called \emph{magnitude} and \emph{Euclidean norm}) of
a discrete signal $\mathbf{x}$ is defined by
\begin{equation}
  ||\mathbf{x}||_2 = \sqrt{\sum_i\mathbf{x}_i^2}.
\end{equation}
Notice that the L2 norm and the Mean Square Error (MSE) are closely
related concepts, because
\begin{equation}
  ||\mathbf{x} - \mathbf{y}||_2^2 = N\cdot\text{MSE}(\mathbf{x} - \mathbf{y}),
\end{equation}
where $N$ is the length of $\mathbf{x}$.

%}}}

\section{Uniform distribution}

The uniform distribution, usually denoted by $\mathcal{U}(c)$, is a
continous probability distribution in which all outcomes are equally
likely, and can br described by \gls{PDF}
\begin{equation}
  \Pr(X{=}x)) =
  \begin{cases}
    \frac{1}{b-a} & \text{for}\quad a \le x \le b, \\
    0 & \text{otherwise},
  \end{cases}
\end{equation}
where $X$ represents a continuous random variable, and $[a, b]$ is the
range of amplitudes of the random samples.

As can be seen, the mean
\begin{equation}
  \mu = \frac{a+b}{2},
\end{equation}
and the variance
\begin{equation}
  \sigma^2 = \frac{(b-a)^{2}}{12}.
\end{equation}

\section{Gaussian distribution}
\label{sec:gaussian_distribution}
%{{{

A continuous random variable $X$ that is generated following a Gaussian
distribution (also known as the normal distribution and usually denoted
by $\mathcal{N}(\mu, \sigma^2)$) is described by a \gls{PDF}
\begin{equation}
  \Pr(X{=}x) = \frac{1}{\sqrt{2\pi}\sigma} e^{-\frac{(x-\mu)^2}{2\sigma^2} },
  \label{eq:normal_PDF}
\end{equation}
where $x$ is a random sample of $X$, $\Pr(X{=}x)$ represents the
probability of finding $x$ in $X$, $\mu$ is the mean of $X$, and
$\sigma$ is the standard deviation of $X$.

When we work with a digital signal (quantized discrete random
variable) $\mathbf{x}$, the \gls{PDF} becomes a
\gls{PMF}\footnote{Also called discrete \gls{PDF}.}
\begin{equation}
  \Pr(\mathbf{x}{=}\mathbf{x}_i) = \frac{1}{\sqrt{2\pi}\sigma} e^{-\frac{(\mathbf{x}_i-\mu)^2}{2\sigma^2} },
  \label{eq:normal_PMD}
\end{equation}
where $\mathbf{x}_i$ is the $i$-th sample of $\mathbf{x}$.

%}}}

\section{Poisson distribution}
\label{sec:poisson_distribution}
%{{{

The Poisson distribution, denoted by $\mathcal{P}(\lambda)$ is a
discrete probability distribution that models the number of times an
event occurs in a fixed interval of time or space, under the following
assumptions:
\begin{enumerate}
\item Events occur independently.
\item The average rate (events per interval) is constant.
\item Two events cannot occur at the exact same instant.
\end{enumerate}
Its \gls{PMF} is:
\begin{equation}
  \Pr(\mathbf{x}{=}\mathbf{x}_i) = \frac{e^{-\mathbf{\lambda}}\lambda^{\mathbf{x}_i}}{\mathbf{x}_i!},
  \label{eq:PN}
\end{equation}
where $\lambda$ is the mean and the variance.

%}}}



